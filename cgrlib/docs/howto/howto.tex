% Main latex file for the cgrlib howto.

% The left footer on the normal pagestyle pages
\newcommand{\normalleftfoot}{\textsl{cgrlib howto}}

% The right footer on the normal pagestyle pages
\newcommand{\normalrightfoot}{\today}


% The manual_template controls page layout and things like the logo
% in each page footer.
\input{doctools/latex/manual_template.tex}  
\input{doctools/latex/mycommands.tex} % Some useful commands
\input{doctools/latex/units.tex} % Macros for pretty measurement units

% Some text is inserted in drafts that shouldn't be in the released
% version.  For example, some figures have their filenames printed in
% the draft version.  See the \draftnote command in mycommands to
% insert text like this.
\newcommand{\isdraft}{1} % Choose 1 for draft, 0 for release

% isforme
% Choose 1 for in-house use, 0 for outside.  The in-house manual
% will contain the dollar-sign calibration commands
\newcommand{\isforme}{1} % Choose 1 for internal, 0 for outside

% Figure search path
\graphicspath{
              {figs/} % Figures drawn with xfig
              }

% These commands cause  usr_commands.idx and srs_commands.idx to be
% generated when latex is run on this file and the multind.sty file
% is pulled in (by manual_template)
%% \makeindex{user_cmds_index}
%% \makeindex{internal_cmds_index}

\begin{document}


%\doublespacing %set double spacing

% Title page
\thispagestyle{empty}   % First page does not need headers and footers
\vspace*{\fill}
\begin{center}
{\huge cgrlib howto}\\
\today
\end{center}
\vspace*{\fill}
\clearpage


% Table of contents page
\pagestyle{toc} % Set the toc page style
\tableofcontents{}
\clearpage{}    % There must be a \clearpage to end a page style

\pagestyle{normal} % Set the normal page style

%%%%%%%%%%%%%%%%%%%%%%%%%%%%%%%%%%%%%%%%%%%%%%%%%%%%%%%%%%%%%%%%%%%%%
% The logger
%
% Subsections:
% 
%%%%%%%%%%%%%%%%%%%%%%%%%%%%%%%%%%%%%%%%%%%%%%%%%%%%%%%%%%%%%%%%%%%%%
\include{logger}



%--------------------------------------------------------------------
% Bibliography
%
% 
%--------------------------------------------------------------------
\clearpage
% \addcontentsline{toc}{section}{Bibliography}
% \pagestyle{references} %set the references page style
% \bibliography{doctools/latex/ecrefs.bib}

















%%%%%%%%%%%%%%%%%%%%%%%%%%%%%%%%%%%%%%%%%%%%%%%%%%%%%%%%%%%%%%%%%%%%%%%%%%%
%Everything below here goes in the appendix
%%%%%%%%%%%%%%%%%%%%%%%%%%%%%%%%%%%%%%%%%%%%%%%%%%%%%%%%%%%%%%%%%%%%%%%%%%%
\appendix





%%%%%%%%%%%%%%%%%%%%%%%%%%%%%%%%%%%%%%%%%%%%%%%%%%%%%%%%%%%%%%%%%%%%%%%%%%%
% Indexes
%
%
%%%%%%%%%%%%%%%%%%%%%%%%%%%%%%%%%%%%%%%%%%%%%%%%%%%%%%%%%%%%%%%%%%%%%%%%%%%




% Redefine the printindex command to format the index in multiple columns.
% Uses the multicol package.  
% Don't try to use column separators (lines) -- there's still something
% strange going on with column widths.
%
% printindex still takes two arguments:
% 1. The ind file containing the index entries
% 2. The name for the index (will appear in the table of contents and at the
%    top of the index page.
\renewcommand{\printindex}[2]{
  % The \endtheindex command redefined by multind confounds the multicol
  % environment.  Redefine it to do nothing.  
  \renewcommand{\endtheindex}{}
  \addcontentsline{toc}{section}{#2}
  % Ideally, printindex would take an argument for the number of columns
  % instead of just hard coding it.
  \begin{multicols}{4}
    [\section*{#2}]
    \input{#1.ind}
  \end{multicols}
}% end printindex





% Alphabetical command index
% \printindex{user_cmds_index}{Alphabetical command index}

%Internal SRS commands
\ifthenelse{\equal{\isforme}{1}}{
\clearpage{}
% \printindex{internal_cmds_index}{Internal command index}
}{}



\end{document}
